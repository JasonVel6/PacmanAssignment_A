\documentclass[12pt,a4paper]{article}
\usepackage[utf8]{inputenc}
\usepackage{amsmath}
\usepackage{amsfonts}
\usepackage{amssymb}
\usepackage{amsthm}
\usepackage[english, greek]{babel}

\newtheorem{pro}{Πρόβλημα}
\newtheorem*{sol}{Λύση}
\usepackage{geometry}
\usepackage{graphicx}
\newcommand{\lt}{\latintext}
\newcommand{\gt}{\greektext}


\makeatletter         
\def\@maketitle{
	\vspace*{1cm}
	\begin{center}
		{\Huge \bfseries \sffamily \@title }\\[4ex] 
		{\Large  \@author}\\[4ex] 
		
		\@date\\[5ex]
		\vspace*{0.5cm}
		\includegraphics[width=1\linewidth, height=0.4\paperheight]{large}
\end{center}}
\makeatother

\author{Βαρβαρίγγος Ιωάννης ΑΕΜ: 8782 \\ Βελέντζας Ιάσων-Γεώργιος ΑΕΜ: 8785}
\title{\lt DS-Pacman.PartA}
\date{5ο Εξάμηνο}
\begin{document}
	
	\maketitle
		\thispagestyle{empty}
	\newpage
	\tableofcontents
	\newpage
	\section{Περιγραφή Παιχνιδιού}
\paragraph*{}
 Το παιχνίδι του {\lt Pacman} είναι λίγο έως πολύ γνωστό σε όλους. Σ’ αυτήν την εργασία ασχολούμαστε με μια απλουστευμένη εκδοχή του, στην οποία ο {\lt Pacman} προσπαθεί απλώς να αποφύγει τα φαντάσματα, ενώ τα φαντάσματα προσπαθούν με τη σειρά τους να πιάσουν τον {\lt Pacman}. Ο “ήρωάς” μας νικάει μόνο αν ολοκληρώσει έναν προκαθορισμένο αριθμό κινήσεων πριν τον πιάσει κάποιο φάντασμα. Διαφορετικά, νικητές θεωρούνται τα φαντάσματα. \paragraph*{}
	  Όσον αφορά την προσομοίωση του παιχνιδιού, εκτυλίσσεται σε ένα δωμάτιο - λαβύρινθο που περιβάλλεται περιμετρικά από τοίχους και είναι χωρισμένο σε τετραγωνικά κελιά. Ο {\lt Pacman} αναπαρίσταται ως ένα πράσινο τετραγωνάκι, ενώ τα φαντάσματα ως κόκκινα.
	\section{Περιγραφή Προβλήματος}
	\paragraph*{}
	Ζητούμενο της παρούσας εργασίας είναι η σχεδίαση ενός αλγορίθμου, ο οποίος να καθορίζει την κίνηση των φαντασμάτων, καθώς και η υλοποίηση αυτού του αλγορίθμου σε {\lt Java}. Για το σκοπό αυτό θα τροποποιήσουμε τη συνάρτηση:
	\begin{equation*}
	int\;[ \;] \; \; calculateNextGhostPosition( \; Room[\;][\;] \; Maze,\;int[\;][\;]\; currentPos)
	\end{equation*}
	Οι κανόνες που διέπουν την κίνηση των φαντασμάτων είναι οι εξής:
	\paragraph*{}
	\begin{itemize}
	
		\item Είναι τυχαία
		\item Δεν οδηγεί το φάντασμα σε κελί που έχει τοίχο (σύγκρουση με τοίχο)
		\item Δεν οδηγεί το φάντασμα σε κελί που έχει άλλο φάντασμα ή σε κελί προς το οποίο κινείται άλλο φάντασμα (σύγκρουση με φάντασμα)
	\end{itemize}
\paragraph*{}
	Αν η κίνηση του φαντάσματος οδηγεί σε σύγκρουση, τότε ακυρώνεται η κίνηση αυτή και το φάντασμα παραμένει στη θέση του. 
	Ο καθορισμός της κίνησης γίνεται μέσω της απόδοσης νέας \underline{κατεύθυνσης} σε κάθε φάντασμα.
	\paragraph*{} Οι τιμές κατεύθυνσης  είναι: 

	
				\begin{center}
			
			 \lt West \; = \; 0 \\ 
		 \lt South \; = \; 1 	 \\	
			 \lt East  \; = \; 2 \\
		  \lt North \; = \; 3\\
		
	\end{center}
	
	\section{Σχεδίαση Αλγορίθμου}
	 \paragraph*{}
	Για την επίλυση του προβλήματος σχεδιάστηκε ο παρακάτω αλγόριθμος: \footnote{\emph{ \textbf{Σημείωση:} Για να γίνει πιο κατανοητή η περιγραφή του αλγορίθμου θα ονομάσουμε την κατεύθυνση που δίνεται τυχαία στο φάντασμα (πριν ελεγχθεί αν οδηγεί σε έγκυρη κίνηση) \textbf{{\lt tempdir}} και την τελική κατεύθυνση που θα δοθεί για κάθε φάντασμα \textbf{{\lt newdir}}.}} 
	\begin{enumerate}
		\item 	Για κάθε φάντασμα θέσε μια τυχαία {\lt tempdir}.
		\item 	Θέσε μια μη έγκυρη {\lt newdir}.
		\item 	Επανάλαβε τα βήματα 1-2 για κάθε φάντασμα.
		\item 	Όσο η {\lt newdir} ενός φαντάσματος είναι μη έγκυρη, έλεγχε αν η {\lt tempdir} οδηγεί σε έγκυρη κίνηση.
		\item 	Αν η κίνηση που ορίζει η {\lt tempdir} είναι έγκυρη, θέσε ως {\lt newdir} την {\lt tempdir}.
		Αλλιώς, θέσε μια άλλη {\lt tempdir}	.
		\item 	Επανάλαβε τα βήματα 4-5 για κάθε φάντασμα.
		\item	Επίστρεψε τις τιμές {\lt newdir} όλων τον φαντασμάτων.
	\end{enumerate}
	
	\section{Ανάλυση Αλγορίθμου}\paragraph*{}
	Για την υλοποίηση του αλγορίθμου αποθηκεύουμε τις τιμές των {\lt tempdir} και {\lt newdir} σε δύο μονοδιάστατους πίνακες, των οποίων το μέγεθος είναι ίσο με το πλήθος των φαντασμάτων. 
	\paragraph*{}Το πλήθος των φαντασμάτων με τη σειρά του καθορίζεται από τους κανόνες του παιχνιδιού και ανακτάται μέσω της σταθεράς:	\footnote{Στα πλαίσια του παιχνιδιού έχει αποδοθεί η τιμή 4 μέσω της εντολής:
		\begin{equation*}
		public \; static\; final\; int\; numberOfGhosts=4\; ;
		\end{equation*}}
\begin{equation*}
int \; \; \;  PacmanUtilities.numberOfGhosts
\end{equation*} 	
 
	
	\subsection{Αρχικοποίηση \lt tempdir} \paragraph*{}
	Θέλουμε να αρχικοποιήσουμε τυχαία τη μεταβλητή {\lt tempdir} με έναν ακέραιο αριθμό στο διάστημα $\left[0,3\right]$. Γι' αυτό χρησιμοποιούμε τη συνάρτηση {\lt Math.random()} που δεν δέχεται ορίσματα και επιστρέφει έναν αριθμό τύπου \lt double \gt στο διάστημα [0,1). Επομένως, αν πολλαπλασιάσουμε με 4 τον αριθμό που επιστρέφει η {\lt Math.random()}, παίρνουμε έναν \lt double \gt στο διάστημα [0,4), τον οποίο μπορούμε να μετατρέψουμε σε {\lt int} εφαρμόζοντας {\lt type-casting}. Εν τέλει καταλήγουμε σε μια έγκυρη τιμή κατεύθυνσης, δηλαδή έναν ακέραιο στο διάστημα [0,3]. \paragraph*{} Η γραμμή κώδικα που υλοποιεί την εκχώρηση τιμής που μόλις περιγράψαμε είναι: 
	\begin{equation*}
	tempGhostDirection[i] = (int)\;(Math.random()*4);
	\end{equation*}
	
	\subsection{Αρχικοποίηση \lt newdir}
\paragraph*{}	Δίνεται στη μεταβλητή {\lt newGhostDirection[i]} μια \underline{\emph{μη έγκυρη}} τιμή μέσω μιας απλής εκχώρησης τιμής:
	\begin{equation*}
	newGhostDirection[i] = -1;
	\end{equation*}
\paragraph*{}	Θυμίζουμε ότι έγκυρες τιμές θεωρούνται οι: 0,1,2,3. Ο λόγος της αρχικοποίησης αυτής εξηγείται στο 4ο βήμα.
	
	\subsection{Δημιουργία Βρόχου Αρχικοποίησης}
\paragraph*{}	Τα βήματα 1 και 2 επαναλαμβάνονται για κάθε φάντασμα, όπως ορίζει το βήμα 3. Επομένως υλοποιούνται εντός ενός βρόχου επανάληψης με:
	\begin{itemize}
		\item 	αρχική συνθήκη: μετρητής = 0
		\item 	συνθήκη ελέγχου: μετρητής \lt $< PacmanUtilities.numberOfGhosts$
		\item 	\gt βήμα αύξησης : +1
	\end{itemize}
			\paragraph{\underline{\lt Code Implementation:}}
			
			\begin{equation*}
			for \; (int \; i=0; \;i<PacmanUtilities.numberOfGhosts; \;i++)
\end{equation*}
\paragraph*{}Ο μετρητής συμβολίζεται με τη μεταβλητή \lt int i. \gt
	
	
	\subsection{Υλοποίηση Ελέγχων}
\paragraph*{}	Δημιουργούμε ένα βρόχο επανάληψης ώστε να διεξάγουμε τους ελέγχους για σύγκρουση. Συνθήκη ελέγχου του βρόχου είναι η εγκυρότητα της {\lt newdir}. Σε περίπτωση που η {\lt newdir} έχει έγκυρη τιμή δεν θα εκτελεστεί ο βρόχος. Γι’ αυτό αρχικοποιήσαμε τη μεταβλητή {\lt newdir} με μη έγκυρη τιμή, ώστε να εκτελεστούν τουλάχιστον μια φορά οι έλεγχοι σύγκρουσης.
	
	\subsubsection{Σύγκρουση με Τοίχο}
\paragraph*{}	Για τον έλεγχο σύγκρουσης του \lt i\gt -οστού φαντάσματος με τοίχο χρησιμοποιούμε τον πίνακα 
	\begin{equation*}
	int\; Maze[x][y].\textbf{ \lt walls}[k]
	\end{equation*}
\paragraph*{}	Ο ακέραιος που επιστρέφεται είναι 0 αν υπάρχει τοίχος και 1 αν δεν
	υπάρχει.  \paragraph*{} Πρέπει να εκφράσουμε τις παραμέτρους με βάση ήδη γνωστά δεδομένα και δη τον δισδιάστατο πίνακα \lt currentPos, \gt ο οποίος σε κάθη γραμμή έχει το \lt i \gt-οστό φάντασμα και στις στήλες έχει τετμημένη και τεταγμένη αντίστοιχα άρα:
	
	 \begin{itemize}
		\item	 \lt x: \gt αριθμός γραμμής στην οποία βρίσκεται το φάντασμα (τετμημένη)
		\begin{equation*}
		x[i]= currentPos[i][0];
		\end{equation*}
		\item	 \lt y: \gt αριθμός στήλης στην οποία βρίσκεται το φάντασμα (τεταγμένη)
		\begin{equation*}
		y[i]=currentPos[i][1];
		\end{equation*}
		\item	 \lt k: \gt κατεύθυνση προς την οποία πρόκειται να κινηθεί το φάντασμα
		\begin{equation*}
		k[i]=tempGhostDirection[i];
		\end{equation*}	
	\end{itemize}
\paragraph*{}	Επομένως για να μην υπάρχει τοίχος στην κατεύθυνση \lt k  \gt από το κελί που βρίσκεται το φάντασμά μας πρέπει:
	
	\paragraph{\underline{\lt Code Implementation:}}
	
	\begin{equation*}
		Maze\; [\; currentPos[i][0]\;] \; [\;currentPos[i][1] \; ].walls[\;tempGhostDirection[i]\;] == 1
	\end{equation*}
	
	\subsubsection{Σύγκρουση με Άλλο Φάντασμα}
\paragraph*{}	Για τον έλεγχο αυτό χρησιμοποιούμε τη συνάρτηση:
	\begin{equation*}
	boolean[\;] \; checkCollision(\; int[\;] \; moves, int[\;]\;[\;]\;currentPos\;)
	\end{equation*}
	
\paragraph*{}	Η συνάρτηση αυτή δέχεται ως ορίσματα την {\lt tempdir} του φαντάσματος και τη θέση του στο λαβύρινθο και επιστρέφει \lt true \gt αν η κατεύθυνση του φαντάσματος οδηγεί σε σύγκρουση ή \lt false \gt αν η κατεύθυνση οδηγεί σε έγκυρη κίνηση. \\ Επομένως, χρειαζόμαστε έναν πίνακα ο οποίος θα αποθηκεύει τις τιμές που επιστρέφει η \lt checkCollision.\gt Αυτός είναι ο: 
	
	\begin{equation*}
	boolean[\;] \; ghostCollisionStatus
	\end{equation*}
	
	που υλοποιείται:
	 
	\paragraph{\underline{\lt Code Implementation:}}
	
	\begin{equation*}
	ghostCollisionStatus[i]\; =\; checkCollision(\;tempGhostDirection \;, \; currentPos\;)
	\end{equation*}
	
	Άρα η συνθήκη μας για να μην υπάρχει σύκγρουση με άλλο φάντασμα είναι:
	
	
		\paragraph{\underline{\lt Code Implementation:}}
	
	\begin{equation*}
!\; ghostCollisionStatus[i]
	\end{equation*}
	\begin{center}
		\textbf{Η τελική συνθήκη ελέγχου δίνεται από το λογικό ΚΑΙ των 2 συνθηκών}	
	\end{center}
\paragraph{\underline{\lt Code Implementation:}}
\begin{equation*}
if \; (\;Maze[\;currentPos[i][0]\;]\;[\;currentPos[i][1]\;].walls[\;tempGhostDirection[i]] \;== \;1 
\end{equation*}
\begin{center}
\textbf{\&\&}
\end{center}
\begin{equation*}   !ghostCollisionStatus[i]\; ) 

\end{equation*}
\subsection{Ανάθεση τιμής στην \lt newdir}
\paragraph*{} \gt Με βάση την παραπάνω συνθήκη ελέγχου δημιουργούμε μια δομή \lt if-else \gt, έτσι ώστε όταν η συνθήκη είναι αληθής να εκχωρείται στην \lt newdir \gt η τιμή της \lt tempdir, \gt ενώ όταν είναι ψευδής, η \lt newdir \gt να παραμένει μη έγκυρη (επανεκτελείται λοιπόν, ο βρόχος ελέγχου \lt while) \gt και η \lt tempdir \gt παίρνει μια καινούρια τυχαία τιμή, όπως στο βήμα 1: \\
\paragraph{\underline{\lt Code Implementation:}}
\paragraph{\gt Εκχώρηση τιμής στην \lt newdir:}
\begin{equation*}
newGhostDirection[i] = tempGhostDirection[i];
\end{equation*}

\subsection{Δημιουργία Βρόχου Ελέγχων}
\paragraph*{}Όμοια με το 3ο βήμα: \\
\paragraph*{}Τα βήματα 4 και 5 επαναλαμβάνονται για κάθε φάντασμα. Επομένως υλοποιούνται εντός ενός βρόχου επανάληψης με:
\begin{itemize}
	\item 	αρχική συνθήκη: μετρητής = 0
	\item 	συνθήκη ελέγχου: μετρητής \lt $< PacmanUtilities.numberOfGhosts$
	\item 	\gt βήμα αύξησης : +1
\end{itemize}
\paragraph*{}ο οποίος θα περιέχει το βρόχο \lt while \gt και τη δομή \lt if-else, \gt εμφωλευμένη στο βρόχο \lt while.
\paragraph{\underline{\lt Code Implementation:}}

\begin{equation*}
for \; (int \; i=0; \;i<PacmanUtilities.numberOfGhosts; \;i++)
\end{equation*}
\paragraph*{} \gt Ο μετρητής συμβολίζεται με τη μεταβλητή \lt int i. \gt
\subsection{Επιστροφή Πίνακα Τιμών}
\paragraph*{}Τέλος η συνάρτηση πρέπει να επιστρέφει έναν πίνακα ακεραίων που να καθορίζει την κίνηση των φαντασμάτων. Αυτός είναι ο \lt int[ \;] \;newGhostDirection:
\paragraph{\underline{\lt Code Implementation:}}

\begin{equation*}
  return \; newGhostDirection;
\end{equation*}
\end{document}